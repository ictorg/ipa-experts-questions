\section{Einleitung}

Dieses Formular dient zur Vorbereitung und Durchführung von IPA Fachgesprächen. Ein Fachgespräch dient dazu Themen und Inhalte aus der vorliegenden IPA zu vertiefen, zu prüfen und Unklarheiten auszuräumen.

\begin{itemize}
  \item Das Fachgespräch dauert zwischen 30 und 60 Minuten.
  \item Das Fachgespräch kann auf Schweizerdeutsch, Deutsch oder ggf. Englisch geführt werden.
  \item Acht bis zehn Themen sind vorzubereiten und zu besprechen. Frage die verantwortliche Fachkraft nach Vorschlägen.
  \item Sechs Gesprächsthemen fliessen in die Bewertung der IPA ein.
  \begin{itemize}
    \item Übrige und schlecht beantwortete Themen dürfen gestrichen werden.
    \item Haben zwei Themen dasselbe Überthema (bspw. könnte \enquote{Styling mit CSS} und \enquote{Entwickeln von UI Komponenten} zu \enquote{Entwicklung der Benutzeroberfläche} werden) dürfen diese zusammengeführt werden.
  \end{itemize}
\end{itemize}

Für das Gespräch sollen unbedingt folgende Aspekte beachtet werden:

\begin{itemize}
  \item Im Fachgespräch sollen \textbf{keine} Fragen gestellt werden, welche \textbf{auswendig gelerntes Wissen abfragen}, sondern auf das fachmännische Handeln abzielen, sowie das gezielte Vorgehen des Kandidaten begründen.
  \begin{itemize}
    \item[\ding{51}] Wie sind Sie vorgegangen, um den im Bericht auf Seite 21 gezeigten regulären Ausdruck (Regular Expression, RegEx), zu entwickeln? Wie funktioniert dieser? Wie haben Sie den regulären Ausdruck getestet?
    \item[\ding{55}] Was bedeutet \texttt{\char`\\W} in einem regulären Ausdruck? Von wem wurden reguläre Ausdrücke erfunden? Seit wann gibt es reguläre Ausdrücke? Welche Namen gibt es noch für reguläre Ausdrücke?
    \item[\ding{55}] Welche Schichten gibt es im OSI-Modell? Welche Phasen hat IPERKA? Welche HTML-Elemente kennen Sie? Welche Netzwerkgerätetypen gibt es?
  \end{itemize}
  \item \textbf{Hypothetische oder kontextfremde Fragen} sollten \textbf{vermieden} oder am Ende eines Fragenkomplexes gestellt werden, da damit oft die Gefahr besteht, vom Niveau zu schwierig oder zu umfangreich zu sein.
  \begin{itemize}
    \item[\ding{51}] Was passiert, falls im auf Seite 32 gezeigten Suchfeld die Zeichenkette \texttt{\char`\"\ UNION DROP DATABASE kunden;} eingegeben wird? Wie könnte dieses Problem verhindert werden?
    \item[\ding{55}] Stellen Sie sich vor, die Software wird von einer Hackergruppe angegriffen. Wie schützen Sie die Applikation vor so einem Angriff?
    \item[\ding{55}] Sie haben IPERKA fürs Projektmanagement eingesetzt. Was würde sich verändern, falls Sie HERMES einsetzen müssten? 
  \end{itemize}
  \item Der \textbf{Schwierigkeitsgrad} soll an den angestrebten Abschluss \textbf{angepasst} sein und es soll keine gute Portion Glück notwendig sein, um eine Antwort zu finden.
  \begin{itemize}
    \item[\ding{51}] Um das Passwort nicht als Klartext speichern zu müssen, setzten Sie \enquote{bcrypt} ein. Was tut diese Funktion mit dem eingegebenen Passwort? Wie kann nun überprüft werden, ob der Benutzer bei einem Login das richtige Passwort eingegeben hat?
    \item[\ding{55}] Laut der Aufgabenstellung sollten Sie die Passwörter nicht im Klartext speichern. Dafür haben Sie das Verfahren \enquote{bcrypt} eingesetzt. Was macht \enquote{bcrypt} sicherer als \enquote{md5} und \enquote{sha1}?
    \item[\ding{55}] In den Seiten 56 bis 59 gibt es einen Schreibfehler im Quellcode. Wo ist dieser und wie würden Sie diesen beheben?
  \end{itemize}
\end{itemize}